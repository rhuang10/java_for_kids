%        File: hello_world.tex
%     Created: Sun Jun 13 10:00 am 2021 B
% Last Change: Sun Jun 13 10:00 am 2021 B
%
\documentclass[a4paper]{article}
\usepackage{listings}
\usepackage{color}
\usepackage[T1]{fontenc}
\usepackage{textcomp}
\usepackage{fancybox, framed}
\definecolor{mygreen}{rgb}{0,0.6,0}
\definecolor{mygray}{rgb}{0.5,0.5,0.5}
\definecolor{mymauve}{rgb}{0.58,0,0.82}
\definecolor{lightgray}{rgb}{0.9,0.9,0.9}
\linespread{1.25}
\lstset{ 
  backgroundcolor=\color{white},   % choose the background color; you must add \usepackage{color} or \usepackage{xcolor}; should come as last argument
  basicstyle=\footnotesize,        % the size of the fonts that are used for the code
  breakatwhitespace=false,         % sets if automatic breaks should only happen at whitespace
  breaklines=true,                 % sets automatic line breaking
  captionpos=t,                    % sets the caption-position to bottom
  commentstyle=\color{mygreen},    % comment style
  deletekeywords={...},            % if you want to delete keywords from the given language
  escapeinside={\%*}{*)},          % if you want to add LaTeX within your code
  extendedchars=true,              % lets you use non-ASCII characters; for 8-bits encodings only, does not work with UTF-8
  firstnumber=1,                % start line enumeration with line 1000
  keepspaces=true,                 % keeps spaces in text, useful for keeping indentation of code (possibly needs columns=flexible)
  keywordstyle=\color{blue},       % keyword style
  language=Octave,                 % the language of the code
  morekeywords={*,...},            % if you want to add more keywords to the set
  numbers=left,                    % where to put the line-numbers; possible values are (none, left, right)
  numbersep=5pt,                   % how far the line-numbers are from the code
  numberstyle=\tiny\color{mygray}, % the style that is used for the line-numbers
  rulecolor=\color{black},         % if not set, the frame-color may be changed on line-breaks within not-black text (e.g. comments (green here))
  showspaces=false,                % show spaces everywhere adding particular underscores; it overrides 'showstringspaces'
  showstringspaces=false,          % underline spaces within strings only
  showtabs=false,                  % show tabs within strings adding particular underscores
  stepnumber=1,                    % the step between two line-numbers. If it's 1, each line will be numbered
  stringstyle=\color{mymauve},     % string literal style
  tabsize=2,	                   % sets default tabsize to 2 spaces
  upquote=true,
  backgroundcolor=\color{lightgray}
}

\newcommand{\pythonfile}[1]{\lstinputlisting[language=Python, title=Python]{#1}}
\newcommand{\javafile}[1]{\lstinputlisting[language=Java, title=Java]{#1}}

\begin{document}
Let us begin with the simple \verb|hello world|.
\pythonfile{../code/helloworld.py}
The python code is a little bit ``unusual'', compared to what you have typically done.  We define a function \verb|main| and then call it if the script is run. 

\javafile{../code/HelloWorld.java}

Java vs Python
\begin{enumerate}
    \item Python uses \verb|:| and indent for defining a code block. Java uses \verb|{. . .}|.
    \item Python recognises line break as end of the statement. Java uses \verb|;| to end the statement.
\end{enumerate}
Conceptually, you can write the \verb|HelloWorld.java| as
\begin{lstlisting}[language=Java, numbers=none]
class HelloWorld{public static void main(String[] args){System.out.println("Hello world");}}
\end{lstlisting}
But really? It is too hard to understand, especially when the code becomes longer and longer.

A more substle and more important between Java and Python is that, Java is pure object-oriented. Evil Java king does not allow the verb slavers (\verb|method|s) show in any public domain without a noun master (\verb|class|). Therefore, 
\begin{lstlisting}[language=Java, numbers=none]
class HelloWorld{. . .
}
\end{lstlisting}
is required for Java to compile the code, though it looks like useless.


\doublebox{%
\begin{minipage}{\textwidth}
   
    \textbf{if \_\_name\_\_ == " \_\_main\_\_"} \newline

Python uses file name as module name. Therefore, one can reuse the function \verb|main| by
\begin{lstlisting}[language=Python, numbers=none]
import helloworld
helloworld.main()
\end{lstlisting}

When Python does \verb|import|, it runs all the scripts define in \verb|hello_world.py| by default. If we code the \verb|helloworld.py| like the following,
\begin{lstlisting}[language=Python, numbers=none]
def main():
    # do something
main()
\end{lstlisting}
\verb|main()| in the last line would run when anyone does \verb|import helloworld|. In most cases, this is not what we want. \verb|if __name__ == '__main__'| prevents this from happening.
\end{minipage}%
}

\end{document}


