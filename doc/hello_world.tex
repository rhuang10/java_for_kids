%        File: hello_world.tex
%     Created: Sun Jun 13 10:00 am 2021 B
% Last Change: Sun Jun 13 10:00 am 2021 B
%
\documentclass[a4paper]{article}
\usepackage{listings}
\usepackage{color}
\usepackage[T1]{fontenc}
\usepackage{textcomp}
\usepackage{fancybox, framed}
\definecolor{mygreen}{rgb}{0,0.6,0}
\definecolor{mygray}{rgb}{0.5,0.5,0.5}
\definecolor{mymauve}{rgb}{0.58,0,0.82}
\definecolor{myblue}{rgb}{0.68,0.85,0.90}
\definecolor{myyellow}{rgb}{1,1,0.87}
\definecolor{lightgray}{rgb}{0.9,0.9,0.9}
\linespread{1.25}
\usepackage{tabularx}
\usepackage{tablestyles}
\lstset{ 
  backgroundcolor=\color{white},   % choose the background color; you must add \usepackage{color} or \usepackage{xcolor}; should come as last argument
  basicstyle=\footnotesize,        % the size of the fonts that are used for the code
  breakatwhitespace=false,         % sets if automatic breaks should only happen at whitespace
  breaklines=true,                 % sets automatic line breaking
  captionpos=t,                    % sets the caption-position to bottom
  commentstyle=\color{mygreen},    % comment style
  deletekeywords={...},            % if you want to delete keywords from the given language
  escapeinside={\%*}{*)},          % if you want to add LaTeX within your code
  extendedchars=true,              % lets you use non-ASCII characters; for 8-bits encodings only, does not work with UTF-8
  firstnumber=1,                % start line enumeration with line 1000
  keepspaces=true,                 % keeps spaces in text, useful for keeping indentation of code (possibly needs columns=flexible)
  keywordstyle=\color{red},       % keyword style
  morekeywords={*,...},            % if you want to add more keywords to the set
  numbers=left,                    % where to put the line-numbers; possible values are (none, left, right)
  numbersep=5pt,                   % how far the line-numbers are from the code
  numberstyle=\tiny\color{mygray}, % the style that is used for the line-numbers
  rulecolor=\color{black},         % if not set, the frame-color may be changed on line-breaks within not-black text (e.g. comments (green here))
  showspaces=false,                % show spaces everywhere adding particular underscores; it overrides 'showstringspaces'
  showstringspaces=false,          % underline spaces within strings only
  showtabs=false,                  % show tabs within strings adding particular underscores
  stepnumber=1,                    % the step between two line-numbers. If it's 1, each line will be numbered
  stringstyle=\color{mymauve},     % string literal style
  tabsize=2,	                   % sets default tabsize to 2 spaces
  upquote=true
}

\newcommand{\pythonfile}[1]{\lstinputlisting[language=Python,backgroundcolor=\color{myblue}]{#1}}
\newcommand{\javafile}[1]{\lstinputlisting[language=Java, backgroundcolor=\color{myyellow}]{#1}}
\newcommand{\jinline}[1]{\lstinline[language=Java, backgroundcolor=\color{myyellow}]{#1}}
\newcommand{\pyinline}[1]{\lstinline[language=Python, backgroundcolor=\color{myblue}]{#1}}
\lstnewenvironment{jcode}[0]{ \lstset{language=Java,numbers=none,backgroundcolor=\color{myyellow}}}{}
\lstnewenvironment{pycode}[0]{ \lstset{language=Python,numbers=none,backgroundcolor=\color{myblue}}}{}
\lstnewenvironment{tcode}[0]{ \lstset{language=Bash,numbers=none, backgroundcolor=\color{lightgray}}}{}

\newtheorem{theorem}{Theorem}[section]
\newtheorem{exercise}[theorem]{Exercise}

\newcommand{\exinline}[1]{(\refstepcounter{theorem}Exercise~\thetheorem\label{#1})}
\begin{document} 
\section{Hello World} 
Let us begin with the simple \verb|hello world|.
\pythonfile{../code/helloworld.py}
The python code is a little bit ``unusual'', compared to what you have typically done.  We define a function \verb|main| and then call it if the script is run. 

\javafile{../code/HelloWorld.java}

Java vs Python
\begin{enumerate}
    \item Python uses \verb|:| and indent for defining a code block. Java uses \verb|{. . .}|.
    \item Python recognises line break as end of the statement. Java uses \verb|;| to end the statement.
\end{enumerate}
Conceptually, you can write the \verb|HelloWorld.java| as
\begin{jcode}
class HelloWorld{public static void main(String[] args){System.out.println("Hello world");}}
\end{jcode}
But really? It is too hard to understand, especially when the code becomes longer and longer.

A more substle and more important between Java and Python is that, Java is pure object-oriented. Evil Java king does not allow the verb slavers (\verb|method|s) show in any public domain without a noun master (\verb|class|). Therefore, 
\begin{jcode}
class HelloWorld{. . .
}
\end{jcode}
is required for Java to compile the code, though it looks like useless.


\doublebox{%
\begin{minipage}{\textwidth}
   
    \lstinline[language=Python]{if __name__ == "__main__"} \newline

Python uses file name as module name. Therefore, one can reuse the function \verb|main| by
\begin{pycode}
import helloworld
helloworld.main()
\end{pycode}

When Python does \verb|import|, it runs all the scripts define in \verb|hello_world.py| by default. If we code the \verb|helloworld.py| like the following,
\begin{pycode}
def main():
    # do something
main()
\end{pycode}
\verb|main()| in the last line would run when anyone does \verb|import helloworld|. In most cases, this is not what we want. \verb|if __name__ == '__main__'| prevents this from happening.
\end{minipage}%
}

Run python code in a terminal,
\begin{tcode}
  python hello_world.py  
\end{tcode}

Run java code in a terminal
\begin{enumerate}
    \item Compile the Java code
        \begin{tcode}
        $ javac HelloWorld.java
        \end{tcode}
    \item Check the compilation result
        \begin{tcode}
        $ ls
        \end{tcode}
        You will see there is a new file name \verb|HelloWorld.class| showing up.
    \item Run program
        \begin{tcode}
        $ java HelloWorld
        \end{tcode}
\end{enumerate}

Why compilation language?
\begin{itemize}
    \item It is typically faster than script language.
\end{itemize}

\begin{exercise}
    \label{ex:java.hello}
    Change the \verb|HelloWorld| java code, line 5 to
    \begin{jcode}
        System.out.println("Hello Sophie!");
    \end{jcode}
    Compile and run the java code in a terminal.
\end{exercise}

\section{Data Types}
\begin{table}
    \tablestyle[sansbold]
    \begin{tabular}{llp{0.8\textwidth}}
        \theadstart
    Data Type&Size&Description\\
    \tbody
byte&1 byte&Stores whole numbers from -128 to 127\\
short&2 bytes&Stores whole numbers from -32,768 to 32,767\\
int&4 bytes&Stores whole numbers from -2,147,483,648 to 2,147,483,647\\
long&8 bytes&Stores whole numbers from -9,223,372,036,854,775,808 to 9,223,372,036,854,775,807\\
float&4 bytes&Stores fractional numbers. Sufficient for storing 6 to 7 decimal digits\\
double&8 bytes&Stores fractional numbers. Sufficient for storing 15 decimal digits\\
boolean&1 bit&Stores true or false values\\
char&2 bytes&Stores a single character/letter or ASCII values\\
\tend
    \end{tabular}
    \caption{Java Primitive data types}
    \label{tab:j.prim.type}
\end{table}
Java data types are divided two groups:
\begin{itemize}
    \item Primitive data types - includes \jinline{byte, short, int, long, float, double, boolean} and \jinline{char}. Tabel~\ref{tab:j.prim.type} shows the primitive data types and the stored value description.
    \item Non-primitive data types - such as String, Arrays and  Classes (you will learn more about these in a later chapter).
\end{itemize}
The ways that Java treats primitive data types and no-primitive data types are very different. We will discuss this once we are talking about class. 

\javafile{../code/DataTypeExample.java}
\pythonfile{../code/data_type_example.py}
Notice that python also has similar data types: \pyinline{int, float, bool} and \pyinline{str}. 
\begin{itemize}
    \item Java language is designed to enforce type safety. A variable must be declared with a type before you can use it. In most cases, you are not allowed to assign a value with different type to this variable.
    \item Python is dynamically typed language. The type of a variable can be changed by the new assigned value.
\end{itemize}

Why type safty?
\pythonfile{../code/type_safty.py}



\begin{exercise}
    How to declare a variable in Java?   
\end{exercise}

\begin{exercise}
    \label{ex:java.assign}
    Can I assign a float value to a integer variable in Java? Why?
\end{exercise}

\begin{exercise}
    \label{ex:python.assign}
    Can I assign a float value to a integer variable in Python? Why?
\end{exercise}


\end{document}


